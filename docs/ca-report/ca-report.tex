\documentclass[12pt,oneside]{memoir}

\usepackage{nus-bcomp-fyp-ca-report}

\usepackage{lipsum}

\usepackage[style=apa, backend=biber]{biblatex}
\addbibresource{references.bib}

\title{Benchmarking and Improving OCR System for Southeast Asian Languages}
\author{Qiu Jiasheng, Jason}
\department{Department of Computer Science}
\faculty{School of Computing}
\university{National University of Singapore}
\academicyear{2024/2025}
\projectid{H0792230}
\supervisor{A/P Min-Yen Kan}
\advisor{Tongyao Zhu}

\begin{document}
\frontmatter

\pagestyle{plain}

\makecover

\setcounter{page}{1}

\maketitle

\chapter{Abstract}
\lipsum[1]

\vspace{0.3in}
Subject Descriptors:
\begin{adjustwidth}{0.7in}{}
    I.2.7 Natural Language Processing
\end{adjustwidth}

Keywords:
\begin{adjustwidth}{0.7in}{}
    Optical Character Recognition
\end{adjustwidth}

Implementation Software and Hardware:
\begin{adjustwidth}{0.7in}{}
    Python, Tesseract, EasyOCR
\end{adjustwidth}

\chapter{Acknowledgement}
I would like to thank my supervisor, A/P Kan Min-Yen, and my advisor, Tongyao Zhu, for their invaluable guidance and mentorship. Their encouragement and constructive guidance have been a significant source of inspiration throughout the project.

\listoftables

\tableofcontents

\mainmatter

\chapter{Introduction}
Optical Character Recognition (OCR) is the process of detecting and converting text in a image into a computer-friendly text format (Santos, 2019).

This project aims to answer the following research questions (RQs):

\begin{itemize}
    \item \textbf{RQ1.} How do popular OCR tools perform on Southeast Asian scripts?
    \item \textbf{RQ2.} What specific linguistic and script-related challenges affect OCR accuracy on Southeast Asian languages?
    \item \textbf{RQ3.} How does the choice of OCR tool impact accuracy on Southeast Asian scripts?
\end{itemize}

\chapter{Related Work}
\section{OCR on Low-Resource Languages}
\parencite{ignat-etal-2022}
\section{Benchmarking OCR}

\chapter{Methodology}

\section{Data Collection}

\section{Benchmarking OCR Tools}

\chapter{Results}

\begin{table}[h]
    \centering
    \begin{tabular}{|l|c|c|}
        \hline
        & EasyOCR & Tesseract \\
        \hline
        English & 0.17 & 0.20\\
        Indonesian & 0.20& 0.18\\
        Vietnamese & 0.30& 0.39\\
        Thai & 0.26 & 0.51\\
        \hline
    \end{tabular}
    \caption{Character Error Rate}
\end{table}

\begin{table}[h]
    \centering
    \begin{tabular}{|l|c|c|}
        \hline
        & EasyOCR & Tesseract \\
        \hline
        English & 0.25 & 0.29\\
        Indonesian & 0.27& 0.33\\
        Vietnamese & 0.31& 0.42\\
        Thai & 1.68 & 1.77\\
        \hline
    \end{tabular}
    \caption{Word Error Rate}
\end{table}

\chapter{Future Work}

\printbibliography[title={References}]

\end{document}
