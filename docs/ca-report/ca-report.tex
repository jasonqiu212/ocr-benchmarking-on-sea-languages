\documentclass[12pt,oneside]{memoir}

\usepackage{nus-bcomp-fyp-ca-report}

\usepackage{lipsum}

\usepackage[style=apa, backend=biber]{biblatex}
\addbibresource{references.bib}

\title{Benchmarking and Improving OCR System for Southeast Asian Languages}
\author{Qiu Jiasheng, Jason}
\department{Department of Computer Science}
\faculty{School of Computing}
\university{National University of Singapore}
\academicyear{2024/2025}
\projectid{H0792230}
\supervisor{A/P Min-Yen Kan}
\advisor{Tongyao Zhu}

\begin{document}
\frontmatter

\pagestyle{plain}

\makecover

\setcounter{page}{1}

\maketitle

\chapter{Abstract}
While Optical Character Recognition (OCR) has been widely studied for high-resource languages such as English and Chinese, the efficacy and limitations of OCR models on Southeast Asian (SEA) languages remain largely unexplored.
This study aims to bridge this gap by assessing and improving the performance of OCR technologies on SEA languages.
To achieve this objective, we propose a reusable pipeline to gather SEA-language text from Wikipedia and benchmark popular OCR tools.

\vspace{0.3in}
Subject Descriptors:
\begin{adjustwidth}{0.7in}{}
    I.2.7 Natural Language Processing
\end{adjustwidth}

Keywords:
\begin{adjustwidth}{0.7in}{}
    Optical Character Recognition
\end{adjustwidth}

Implementation Software and Hardware:
\begin{adjustwidth}{0.7in}{}
    Python, Tesseract, EasyOCR
\end{adjustwidth}

\chapter{Acknowledgement}
I would like to thank my supervisor, A/P Kan Min-Yen, and my advisor, Tongyao Zhu, for their invaluable guidance and mentorship. Their encouragement and constructive guidance have been a significant source of inspiration throughout the project.

\listoftables

\tableofcontents

\mainmatter

\chapter{Introduction}
Current research in Natural Language Processing (NLP) is heavily concentrated on 20 of the 7,000 languages in the world \parencite{magueresse-etal-2020}.
In particular, Southeast Asia (SEA) is home to over 1,000 languages, but remains as a relatively under-researched region in NLP \parencite{aji-etal-2023}.
Similar to most low-resource languages, a major challenge in developing NLP systems for SEA languages is the limited availability of datasets for the region’s languages.
Although many scanned documents and books in these low-resource languages are available online, the text within these files remains inaccessible due to formats like images and PDFs.

A solution to this problem is to use Optical Character Recognition (OCR) to extract the textual data.
OCR is the process of identifying and converting text in an image into a computer-friendly text format.
By extracting the text from these scanned documents, OCR can generate valuable datasets for low-resource languages.
The created datasets can then be used for downstream NLP tasks, such as machine translation, training large language models, and POS taggers \parencite{ignat-etal-2022} \parencite{agarwal-and-anastasopoulos-2024}.
Therefore, studying OCR performance on SEA languages is crucial to accelerating NLP research in the region.

While OCR has been widely studied for high-resource languages such as English and Chinese, the efficacy and limitations of OCR models on SEA languages remain largely unexplored.
To address this gap, we propose a reusable pipeline to collect textual data in low-resource SEA languages from Wikipedia and benchmark popular open-source OCR tools on the collected data.
The primary objective is to benchmark and improve the performance of OCR technologies on SEA languages, thereby contributing to the advancement of NLP applications in this linguistically diverse region.
Specifically, this project seeks to answer the following research questions (RQs):

\begin{itemize}
    \item \textbf{RQ1.} How do popular OCR tools perform on SEA scripts?
    \item \textbf{RQ2.} What specific linguistic and script-related challenges affect OCR accuracy on SEA languages?
    \item \textbf{RQ3.} What techniques and recommendations can enhance OCR accuracy on SEA languages?
\end{itemize}

\chapter{Related Work}

\section{OCR on Low-Resource Languages}
Similar to most low-resource languages, a major challenge in developing SEA OCR systems is the limited availability of datasets and benchmarks for the region’s languages.

\parencite{ignat-etal-2022}
\section{Benchmarking OCR}

\section{Improving OCR Accuracy}

\chapter{Methodology}

\section{Data Collection}

\section{Benchmarking OCR Tools}
\parencite{smith-2013}

\chapter{Results}

\begin{table}[h]
    \centering
    \begin{tabular}{|l|c|c|}
        \hline
        & EasyOCR & Tesseract \\
        \hline
        English & 0.17 & 0.20\\
        Indonesian & 0.20& 0.18\\
        Vietnamese & 0.30& 0.39\\
        Thai & 0.26 & 0.51\\
        \hline
    \end{tabular}
    \caption{Character Error Rate}
\end{table}

\begin{table}[h]
    \centering
    \begin{tabular}{|l|c|c|}
        \hline
        & EasyOCR & Tesseract \\
        \hline
        English & 0.25 & 0.29\\
        Indonesian & 0.27& 0.33\\
        Vietnamese & 0.31& 0.42\\
        Thai & 1.68 & 1.77\\
        \hline
    \end{tabular}
    \caption{Word Error Rate}
\end{table}

\chapter{Future Work}

\printbibliography[title={References}]

\end{document}
